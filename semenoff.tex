\input{header.tex}
\usepackage{float}
\begin{document}
%..................................................................
\begin{titlepage}
\par 
\vspace*{-2cm}
\begin{center}
{\sf \Large
\vspace*{1.5cm}
{\Huge Семенов, 10-й семестр, 2018 }\\
{Наконец, философия без истории, а не история философии и не философия истории}}\\
\vspace*{2cm}
\scalebox{1.2}{\includegraphics{thegod.png}} 
\begin{flushright}
\sl\small
by Nyaxx11k\\
\end{flushright}
\begin{flushleft}
\hspace{20pt}\raisebox{-33pt}{\scalebox{0.2}{\includegraphics{bourgeois.png}}}
{\itshape Содержит марксизм}
\end{flushleft}
\end{center}
\end{titlepage}
%..................................................................
\topmargin -1cm 
\hoffset -0.7in 
\textwidth 6.0in 
\textheight 9.0in 
\normalsize 
\pagenumbering{arabic}
%----------------
\tableofcontents
\pagebreak
%----------------

\section{ Философия как наука об истине, теория познания и самый общий метод исследования.}
\section{ Основной вопрос философии.}
\section{ Философия как мировоззрение (онтология). Натурфилософия. Социальная философия (философия истории).}
\section{ Материализм как онтология и как гносеология. Понятие материи.}
\section{ Субъективный идеализм. Солипсизм. Трудности субъективного идеализма.}
\section{ Объективный идеализм. Его онтология и гносеология. Основные компоненты.}
\section{ Агностицизм, или феноменализм.}
\section{ Обыденное и концепциональное знание. Проблема источника концепциального знания. Предлагаемые ее решения.}
\section{ Эмпиризм. Понятия апостериорного и априорного знания.}
\section{ Ступени человеческого познания.}
\section{ Чувственное познание и его формы.}
\section{ Сенсуализм и его основные виды.}
\section{ Проблема природы ощущений и восприятий.}
\section{ Решение проблемы природы ощущений и восприятий естествознанием и материалистической философией.}
\section{ Решение проблемы природы ощущений и восприятий субъективным идеализмом.}
\section{ Кантианское решение проблемы природы ощущений и восприятий.}
\section{ Юмистский ответ на вопрос о природе ощущений и восприятий.}
\section{ Как философы открыли, что мир существует в сознании человека? (элеаты, Демокрит, Т. Гоббс, Дж. Локк, Дж. Беркли, Д. Юм, И. Кант).}
\section{ Что значит знать о чем-то, иметь знание о чем-то?}
\section{ Наивный реализм.}
\section{ Феноменализм. Можно ли его опровергнуть? Можно ли доказать, что мир существует не только в сознании, но и вне него?}
\section{ Проблема отношения вещей в себе и вещей для нас. Кантианское и материалистическое ее решение.}
\section{ Материализм о вещах в себе и вещах для нас. Познание как превращение вещей в себе в вещи для нас.}
\section{ Соотношение мира для нас и мира в себе.}
\section{ «Теория символов» Г. Гельмгольца.}
\section{ «Иероглифический» материализм.}
\section{ Знаки и знаковые системы.}
\section{ Два аспекта природы идеального: гносеологический и онтологический. Психофизиологическая проблема.}
\section{ Психофизическая проблема в домарксистской философской мысли.}
\section{ Психофизическая проблема в современной немарксистской философской мысли.}
\section{ Психофизическая проблема в марксистской философской мысли.}
\section{ Учение И.П. Павлова о высшей нервной деятельности. В чем прав и в чем не прав И.П. Павлов.}
\section{ Два вида идеального: биоидеальное и социоидеальное.}
\section{ Основные понятия теории информации. Природа информации.}
\section{ Два вида объективного существования: самобытие и въинобытие.}
\section{ Чувственное познание в свете теории информации.}
\section{ Идеальное в чувственном познании.}
\section{ Проблема пространства и времени. Кантианское и материалистическое ее решения.}
\section{ Субстанциальная и реляционная концепции времени и пространства}
\section{ Понятие метафизики. Эссенциализм.}
\section{ Открытие умозрения и умозримого (милетцы, элеаты, Демокрит).}
\section{ Виды эссенциализма.}
\section{ Человек как единство духа и тела. Сократ о соотношении духа (души) и тела.}
\section{ Понятие духа. Дух как единство познания и воли.}
\section{ Р. Декарт о соотношении духа (души) и тела. Рефлексы и волевые действия.}
\section{ Чувственное познание как реакции организма (тела).}
\section{ Мышление как волевая деятельность человека.}
\section{ Проблема правильного и неправильного образов действия.}
\section{ Проблема правильного образа (метода) мысли.}
\section{ Философия как метод мышления и наука о мышлении (логика).}
\section{ Два вида мышления: рассудочное и разумное. Две науки о мышлении: формальная логика и логика философская содержательная).}
\section{ Формальная логика — наука о мышлении как субъективной деятельности человека.}
\section{ Основные формы рассудочного мышления.}
\section{ Понятие. Его содержание и объем.}
\section{ Суждение.}
\section{ Умозаключение.}
\section{ Дедукция и индукция.}
\section{ Основные законы формальной логики.}
\section{ Что дает и чего не может дать формальная логика?}
\section{ Современная (символическая) формальная логика.}
\section{ Философия — наука о мышлении как объективном процессе.}
\section{ Открытие Г. Гегелем мышления как объективного процесса и его законов.}
\section{ Конкретно-реальные и логические процессы.}
\section{ Частнологические процессы и вселогический (панлогический) процесс.}
\section{ Основные части философии Г. Гегеля. Сущность гегелевской абсолютной идеи.}
\section{ Понятие диалектики. Основные значения слова.}
\section{ Отличие диалектики Г. Гегеля от диалектики К. Маркса.}
\section{ Философия как наука о мышлении и самый общий метод мышления.}
\section{ Мышление и язык. Понятие и слово.}
\section{ Понятие знака. Треугольник Фреге: знак, предметное значение знака (денотат), смысловое значение знака (десигнат).}
\section{ Проблема природы понятий.}
\section{ Проблема общего и отдельного.}
\section{ Номинализм: крайний и умеренный (концептуализм).}
\section{ Открытие Платоном понятий и мышления как объективного явления.}
\section{ Реализм. Учение Платона о двух мирах.}
\section{ Три решения вопроса об отношении универсалий и вещей.}
\section{ Два вида реализма крайний и умеренный.}
\section{ Материалистический универсизм, или объектализм.}
\section{ Диалектика общего и отдельного.}
\section{ Понятия как образы внешнего мира. Два вида бытия общего: в мире и в мышлении.}
\section{ Понятия как продукты человеческого творчества. Роль фантазии в познании общего.}
\section{ Проблема соотношения чувственного познания и мышления в истории философской мысли.}
\section{ Дает ли мышление новое знание? Проблема и предлагаемые ее решения.}
\section{ Сенсуализм в философии Нового времени и проблема умозрения, познания общего и сущности.}
\section{ Рационализм в философии Нового времени.}
\section{ Познание общего и сущности, и проблема активности мышления}
\section{ Проблема источника гносеологической активности мышления.}
\section{ Кризис материализма в конце XVIII — начале XIX вв,}
\section{ Взгляды французских материалистов XVIII в. на общество. Их социоисторический идеализм и волюнтаризм.}
\section{ Историософская и историческая мысль в поисках объективного источника общественных идей, основы общества и движущих сил истории.}
\section{ Великая французская революция и крах волюнтаризма.}
\section{ Философия истории Г. Гегеля: ее историческое значение.}
\section{ Открытия французских историков эпохи Реставрации.}
\section{ Вклад политической экономии в решение проблемы объективного источника общественных идей, основы общества и движущих сил истории.}
\section{ Философия истории А. Сен-Симона.}
\section{ Экономический детерминизм Р. Джонса.}
\section{ Открытие объективного социального бытия. Возникновение материалистического понимания истории и всеобъемлющего материализма (панматериализма).}
\section{ Открытие материального источника активности мышления.}
\section{ Практика как основа проникновения общего из мира в мышление. Диалектический сенсуализм.}
\section{ Соотношение чувствозримого и умозримого миров для нас.}
\section{ Чувствоумозримый мир в себе и для нас. Создание чувствоумозримого мира для нас как воспроизведение объективного мира.}
\section{ Понятие факта.}
\section{ Методы добывания фактов в естествознании: наблюдение, измерение, эксперимент.}
\section{ Первичная обработка фактов. Единичные и общие факты.}
\section{ Объективность и субъективность фактов.}
\section{ Проблема понимания и объяснения фактов.}
\section{ Интерпретация (истолкование) фактов как их объединение (унитаризация)}
\section{ Два вида унитаризации фактов: холизация и эссенциализация.}
\section{ Теория и теоротекст. Переход от теории к теоротексту (текстуализация) и от теоротексту к теории (интеллектуализация)}
\section{ Идея — ключевое понятие разумного мышления. .}
\section{ Проблема и идея. Практическая и познавательная идея.}
\section{ Рождение идеи. Озарение. Интуиция.}
\section{ Проблемно-фактуальная интуиция.}
\section{ Житейская бессознательная и сознательная холизация. }
\section{ Профессионально-специализированная (расследовательская) холизация.}
\section{ Холизация в исторической науке. Способы проверка холии.}
\section{ Проблема отношения эмпирического и теоретического знания в философии и науке Нового времени.}
\section{ «Философия науки» неопозитивизма и постпозитивизма }
\section{ Факты, эссенциальная идея, гипотеза и теория.}
\section{ Способы проверка гипотезы (теории).}
\section{ Две компонента теории и два пути развития теоретического знания.}
\section{ Проблема «теоретической нагруженности» фактов.}
\section{ Теоретическая триад: теория, мир как он в теории (теорозримый мир) и теоротекст.}
\section{ Теория, теоротекст и теорозримый мир. Понятие физической реальности.}
\section{ Истинное, истинность, истина. Объективность истины.}
\section{ «Корреспондентская» теория истины. Проблема критерия истины.}
\section{ Концепции очевидности. Авторитарная концепция истины.}
\section{ Концепция общезначимости. Конвенциализм.}
\section{ Концепции простоты (экономии мышления) и когерентности.}
\section{ Прагматическая концепция истины.}
\section{ Современный антиверитизм.}
\section{ Проблема суверенности человеческого познания.}
\section{ Абсолютная и относительная истина.}
\section{ Конкретность истины.}
\section{ Граничность (предельность) истины.}
\section{ Диалектика истинного и ложного.}
\section{ Истинное и ложное в развитии мировой философской мысли. Ложноистинные концепции.}
\section{ Иллюзии как вид заблуждения. Идеологические иллюзии. Компоненты идеологии. Партийность идеологии. Объективное в идеологических иллюзиях.}
\section{ Роль идеологии в классовом обществе. Агитация, пропаганда, «пиар».}
\section{ Наука и партийность.}
\section{ Целепланирующая и волевая активности сознания.}
\section{ Проблема существования объективной предопределенности.}
\section{ Проблема свободы и необходимости как основной вопрос философии.}
\section{ Телеология.}
\section{ Возникновение абсолютного детерминизма.}
\section{ Абсолютный детерминизм в философии и естествознании Нового времени.}
\section{ Индетерминизм и свобода.}
\section{ Попытки решения проблемы свободы и необходимости: Эпикур, стоики, Спиноза, Лейьниц, Фихте, Шеллинг.}
\section{ Г. Гегель о случайности и необходимости.}
\section{ Причинность как момент объективной предопределенности. Ее место во всеобщей связи мира. Абсолютизация причинной связи как основа абсолютного детерминизма.}
\section{ Причина и условия. Кондиционализм.}
\section{ Создание вероятностного (относительного, диалектического) детерминизма.}
\section{ Объективный процесс как единство уровня явления и уровня сущности.}
\section{ Возможность и действительность. Возможность и невозможность.}
\section{ Вероятность как единство предопределенности и неопределенности. Вероятность и неизбежность.}
\section{ Предопределенность и неопределенность. Необходимость и случайность.}
\section{ Закономерность как категория вероятностного (вариативного, относительного, диалектического) детерминизма.}
\section{ Проблема детерминизма и современная наука. Квантовая механика, синергетика, тория хаоса.}
\section{ Решение проблемы свободы и необходимости диалектическим материализмом.}
\section{ Две сферы человеческой деятельности: свободная и несвободная, зависимая.}
\section{ Свобода воли: формальная и реальная.}
\section{ Диалектико-материалистическое решение основного вопроса философии.}
\section{ Два отношения сознания и мира: познавательное и практическое. Превращение материального в идеальное и идеального в материальное. }


\end{document}